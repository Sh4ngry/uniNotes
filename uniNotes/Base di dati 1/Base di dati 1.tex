\documentclass{article}
\usepackage{hyperref,graphicx}
\usepackage[margin=3cm]{geometry}
\usepackage[most]{tcolorbox}
\title{Base di Dati 1}
\author{Riccardo Cara}

\begin{document}
\maketitle

\tableofcontents
\newpage

\section{Introduzione ai database e ai DBMS}

    Un \textit{database} o Base di Dati è un insieme di dati organizzati in \textit{strutture dati}, esse ne facilitano la creazione, l'accesso e l'aggiornamento.
    I dati possono essere:

    \begin{itemize}
        \item \textbf{dati strutturati}:
        oggetti rappresentati da piccole stringhe di simboli o numeri

        \item \textbf{dati non strutturati:}
        testo scritto in linguaggio umano
        \end{itemize}

    I dati all'interno di un Database vengono gestiti tramite un Database Management System o DBSM ossia il software per la gestione di grandi masse di dati condivise, strutturate e processabili, residenti su una memoria.

    \paragraph{Informazione strutturata} la struttura dell'informazione dipende dal suo utilizzo e può essere modificata nel tempo, facendo un esempio, se si volessero archiviare i dati di una persona, avremmo:
    \begin{itemize}
        \item{nome}
        \item{cognome}
        \item{data di nascita}
        \item{luogo di nascita}
    \end{itemize}
    L'obbiettivo è di facilitare l'elaborazione dei dati sulla base delle loro relazioni che vengono rappresentate tramite record\label{word:record}:

    \begin{center}
    \begin{tabular}{|c|c|c|c|c|}
        \hline 
        \textbf{nome} & \textbf{cognome} & \textbf{data di nascita}& \textbf{luogo di nascita} \\
        \hline
        Francesco & Totti & 27/09/1976& Roma\\
        \hline
    \end{tabular}
    \end{center}
    per accedere ai dati individuali nella struttura dati, vengono fatte delle "interrogazioni" o \textit{query}. Un database è una risorsa condivisa tra diverse componenti in un organizzazione, quando più componenti vogliono accedere allo stesso dato contemporaneamente si crea un problema di \hyperref[sec:concorrenza]{concorrenza} che comporta la necessità di gestire gli accessi contemporanei agli stessi dati.
\section{Sistema Informativo}
    Un \textit{sistema informativo} è concettualmente organizzato in aggregati di informazioni omogenee (file \footnote{esistono alcuni sottofile chiamati indici che permettono di recuperare velocemente le informazioni dai file principali}), gli aggregati sono le componenti del sistema informativo. Un operazione di aggiornamento ha per oggetto un singolo aggregato, mentre una query può coinvolgerne molteplici.

    \subsection{Livelli di astrazione}
        Un database ha tre livelli di astrazione:
        %
        \begin{itemize}
            \item \textbf{schema esterno}:
            descrizione di una porzione del database tramite un modello logico che può differire dallo schema logico e rispecchia i bisogni e privilegi di accesso di diversi tipi di utente, l'accesso al database avviene solamente tramite lo schema esterno

            \item \textbf{schema logico}:
            descrizione dell'intero database nel "principale" modello logico del DBMS

            \item \textbf{schema fisico}:
            rappresentazione dello schema logico sulle strutture di archiviazione fisica
        \end{itemize}
        %
        quando si parla di schema e istanze, con schema si intende in una tabella, la prima riga che definisce il tipo di dato
        %
            \begin{tabular}{|c|c|c|}
            \hline
                nome &cognome& data di nascita \\
            \hline
            \end{tabular}
        %
        con istanza si intende invece il "corpo" della tabella ovvero i valori correnti.
        %
    \subsection{Linguaggi per database}
        Esistono diversi linguaggi utilizzabili per creare/gestire database, DDL,DML,SQL.
        %
        \begin{itemize}
            \item \textbf{DDL}(Data Definition Language):
            per la definizioni di schemio logici, esterni, fisici e altre operazioni generali

            \item \textbf{DML}(Data Manipulation Language):
            per effettuare query e aggiornare le istanze

            \item \textbf{SQL}(Structured Query Language):
            è un linguaggio standardizzato per database basati su modelli relazionali che permette di fare entrambe le azioni del DDL e del DML
        \end{itemize}
        %
    \subsection{Modello di dati}
        Un modello è una struttura da utilizzare per organizzare i dati di interesse e le loro relazioni, un modello è composto da costruttori di tipo, ad esempio il modello relazionale prevede il costruttore di relazione (organizza in \hyperref[word:record]{record} i dati con una relazione). I modelli dei dati si dividono in due gruppi:
        %
        \begin{itemize}
            \item \textbf{modelli logici:}
            i modelli logici non dipendono da come vengono archiviati i dati fisicamente, ma disponibili secondo il modello tramite i DBMS.
            alcuni modelli logici sono: reticolare, gerarchico, relazionale, a oggetti.
            
            \item \textbf{modelli concettuali:}
            i modelli concettuali sono utilizzati all'inizio della progettazione di un database, essi non sono dipendenti dalle modalità di realizzazione, hanno lo scopo di rappresentare entità del mondo reale e le loro relazioni nelle prime fasi di pianificazione di un database, uno dei più utilizzati è il modello entità-relazione.
        \end{itemize}
        %
        vediamo tipi diversi di modelli dati:
        %
        \begin{itemize}

            \item \textbf{Modello Mesh}:
            Nel modello Mesh, i dati sono rappresentati in collezioni di record omogenei, le relazioni binarie vengono rappresentate come links, implementati come puntatori e quindi dipendenti dalla struttura fisica del database. È possibile rappresentare il modello graficamente:
            
            \includegraphics[]{immagini/modello Mesh.png}

            Il modello è rappresentato da nodi e linee. I nodi rappresentano i record e le linee che connettono i nodi rappresentano i links tra essi. Il modello più popolare di modello mesh è \textit{CODASYL}

            \item \textbf{Modello gerarchico}:
            il modello gerarchico è un tipo di modello mesh in cui i collegamenti fra i nodi vengono fatti in maniera gerarchica, i nodi hanno quindi forma di albero e hanno come padre un solo nodo
            \includegraphics[]{immagini/modello gerarchico.png}

            \item \textbf{modello relazionale}:
            i dati e le relazioni sono rappresentate da valori, non ci sono riferimenti espliciti come i puntatori nei modelli mesh o gerarchici. Gli oggetti del mondo reale sono rappresentati come record in cui i campi sono sono riempiti da informazioni di interesse (nome, cognome, data di nascita).

            \item \textbf{modello a oggetti}:
            è un modello basato su oggetti (persona, macchina, animale), gli attributi descrivono lo stato dell'oggetto
        \end{itemize}
        %
\section{Modello relazionale}
    È già stato definito il Modello relazionale nella precedente sezione, si vuole ora approfondire l'argomento.
    Il modello relazionale si basa sulla nozione matematica di relazione. Si definisce con \textit{dominio} (D) un insieme di valori, un sottoinsieme del prodotto cartesiano di k domini $D_1\times D_2 \times ...\times D_k$ ha nome di \textit{relazione matematica}, introduciamo due definizioni.
    
    \begin{itemize}
        \item \textbf{grado}:
            con grado k si intende la relazione matematica di k domini.

        \item \textbf{attributo}:
            un attributo è definito da un nome A e dal suo dominio dom(A)

            \begin{tabular}{|c|}
                \hline
                \textbf{nome}\\
                \hline
                Nerone\\
                \hline

            \end{tabular}
            
            in cui $A=nome$, $dom(A)=str$ e l'attributo è l'insieme di questi.
        
        \item \textbf{schema}:
            Un insieme di attributi è detto schema, lo schema relazionale viene scritto come $R(A_1,A_2,...,A_k)$, descrive la sua struttura ed è invariante nel tempo.
        
        \item \textbf{istanza}:
            l'instanza $r$ contiene tutti i valori correnti in una relazione $R$. 

        \item \textbf{n-upla}:
            una n-upla è un elemento della relazione (un elemento è la riga di una tabella), il numero di n-uple di una relazione è chiamata cardinalità. In ogni n-upla della relazione di ordine k, i componenti della n-upla sono ordinati secondo k, dati $D_1$,$D_2$,...,$D_k$, gli attributi saranno:

            \begin{tabular}{|c |c |c |c|}
                \hline
                $attributo_1$&$attributo_2$&$...$&$attributo_k$\\
                \hline
            \end{tabular}
            
            una n-upla da un punto di vista matematico, è una funzione definita in $R$(relazione) che associa ogni attributo A in R ad un elemento del dominio $dom(a)$, il valore dell'attributo A equivale a $t[A]$ in cui t è tupla/n-upla.

        \item \textbf{tabella}:
            una tabella è l'insieme delle n-uple di una relazione.
        \item \textbf{schema del database}:
            lo schema del database è un insieme di schemi relazionali. Lo schema di un database relazionale è un insieme $R_1,R_2,...,R_n$ degli schemi. Un database relazionale è l'insieme $r_1,r_2,...,r_n$ in cui $r_1$ è l'istanza di $R_1$.
            
            \paragraph{esempio}
                schema: info città (città,regione,popolazione)
                
                istanza:
                \begin{tabular}{|c|c|c|}
                    \hline
                    Roma & Lazio & 3000000\\
                    \hline
                    Milano &Lombardia & 1500000\\
                    \hline
                    Genova & Liguria& 800000\\
                    \hline
                    Pisa & Toscana & 150000\\
                    \hline
                \end{tabular}
        
        
    \end{itemize}

    \paragraph{esempio}:
        definendo $k=2$ e $D_1=\{bianco,nero\},D_2=\{0,1,2\}$, il prodotto cartesiano sarà $D_1\times D_2=\{(bianco,0),(bianco,1),(bianco,2),(nero,0),(nero,1),(nero,2)\}$, è possibile ora definire una relazione, un esempio di relazione è $\{(bianco,0),(bianco,2),(nero,1)\}$ ovvero una relazione di grado 2 con cardinalità 3.
        Una tupla ovvero $t\in r$ è invece (bianco,0), t[1] rappresenta la stringa "bianco"

    \begin{tcolorbox}[colback= green!10!white, colframe= green!40!black, title=Nota bene]

        non è detto che ogni attributo abbia un valore, ad esempio, in un database degli studenti potrebbe essere previsto il numero di telefono, ma non è detto che ogni studente abbia un cellulare, per questo viene utilizzato il valore Null, il valore Null non corrisponde ad uno 0, non appartiene ad un dominio, se sullo stesso dominio ci sono più Null, non sono lo stesso valore.
    \end{tcolorbox}

    \subsection{vincoli di integrità}
        In un database, ad esempio:

        \begin{table}[h]
            \centering

            \begin{tabular}{|l|l|l|l|}
                \hline
                    matricola & nome & anno iscrizione & dipartimento \\ \hline
                    1028 & Carlo magno & 1680 & A02 \\ \hline
                    1029 & Lucio Dalla & 2003 & B01 \\ \hline
                    1029 & Dario Lampa & 2015 & B08 \\ \hline
            \end{tabular}

        \hfill

            \begin{tabular}{|l|l|l|l|}
                \hline
                    dipartimento &numero dipartimento \\ \hline
                    storia& A02 \\ \hline
                    chimica& B01 \\ \hline
                    letterea& B03 \\ \hline
            \end{tabular}

        \end{table}
        %
        possiamo notare che:
        \begin{itemize}
            \item due persone hanno la stessa matricola
            \item una persona è iscritta dal 1680
            \item il dipartimento B08 non esiste
        \end{itemize}
        %
        per evitare questi problemi si usano dei vincoli di integrità.

        \begin{tcolorbox}[colback= green!10!white, colframe= green!40!black, title=vincoli di integrità]

            un vincolo di integrità è un un vincolo che ogni istanza del database deve rispettare per essere considerato corretto.
            un vincolo può essere:
            \begin{itemize}
                \item \textbf{intrarelazionale}
                    un vincolo definito all'interno di una relazione
                
                \item \textbf{interrelazionale}
                    un vincolo definito su più relazioni
            \end{itemize}

        \end{tcolorbox}

        si introducono i seguenti vincoli:
        \begin{itemize}
            \item \textbf{vincolo di unicità}:
                questo vincolo indica impossibilità avere due tuple con lo stesso valore per un determinato attributo. la matricola deve essere diversa per ogni persona
            \item \textbf{vincolo di esistenza del valore}:
                questo vincolo indica l'impossibilità di avere Null come valore di un attributo specifico. il nome non può essere un Null
            \item \textbf{vincolo di dominio}:
                restrizioni imposte sul dominio di un attributo. anno iscrizione deve essere maggiore di 1970
            \item \textbf{vincolo di tupla}:
                restrizioni imposte su ogni tupla indipendentemente dalle altre
        \end{itemize}
%
    \subsection{chiavi}
        oltre a chiavare, definiamo cosa sono le chiavi.
        \begin{tcolorbox}[colback= green!10!white, colframe= green!40!black, title=chiavi primarie]
            una chiave primaria è un attributo o un insieme di attributi (del minimo numero possibile) che sono diversi per ogni tupla quindi rispettano in vincolo di unicità e ci permettono di riconoscere una singola entità. Le chiavi primarie devono anche rispettare il vincolo di esistenza.
        \end{tcolorbox}
\section{Algebra Relazionale}

\section{Teoria Relazionale}

\section{Organizzazione Fisica}

\section{Concorrenza} \label{sec:concorrenza}


\end{document}
