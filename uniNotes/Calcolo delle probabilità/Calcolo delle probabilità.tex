\documentclass{article}
\usepackage{graphicx, amsmath} % Required for inserting images
\newcommand{\probP}{\text{I\kern-0.15em P}}
\title{Calcolo Delle Probabilità}
\author{Riccardo Cara}
\date{2023/2024}

\begin{document}

\maketitle
\tableofcontents
\newpage
\section{Introduzione}
Iniziamo con un esempio di facile comprensione:
Un anno è composto da 365 giorni, la probabilità di nascere uno specifico giorno è di $\frac{1}{365}$, questa si chiama \textit{probabilità uniforme} poichè ogni giorno dell'anno ha la stessa probabilità.
Se volessimo calcolare la probabilità che due persone siano nate una il giorno x e una il giorno y, dovremmo moltiplicare le probabilità tra loro $\frac{1}{365}*\frac{1}{365}$ ovvero $(\frac{1}{365})^2$, se volessimo sapere la probabilità di un gruppo di n persone di essere nate lo stesso giorno, avremmo $(\frac{1}{365})^n$.
Se volessimo calcolare la probabilità di avere due persone nate lo stesso giorno, prima di tutto definiamo $\Omega$.
$\Omega$, chiamato spazio degli eventi elementari, è l'insieme di tutti i possibili risultati dell'esperimento, la sua cardinalità (numero di elementi) consiste, nell'esempio precedente avente 25 persone, avere tutte le combinazioni di lunghezza 25 di tutti i numeri da 1 a 365 ovvero $|\Omega| = 365^{25}$.
Ora, prendiamo un insieme $A \subset \Omega$ ovvero l'insieme di eventi (date di nascita) in cui 2 persone hanno la stessa data di nascita, definito maticamente come $A=\{\omega \in \Omega |\exists \omega_i,\omega_j \in \Omega | i\neq j\land \omega_i = \omega_j\}$. La probabilità che questo evento accada (due persone hanno la stessa data di nascita) equivale a $\probP(A)=\frac{|A|}{|\Omega|}$.
Vogliamo quindi calcolare esattamente $|A|$, è più facile trovare $|A^c|$ ovvero la cardinalità (numero di elementi nell'insieme) dell'insieme complementare e poi usare la formula $\probP(A)=1-\frac{|A^c|}{|\Omega|}$.
Trovare $A^c$ vuol dire trovare \textbf{tutti i possibili esiti} dell'esperimento, ovvero le combinazioni in cui 25 persone sono nate in giorni diversi (365 giorni). Per fare ciò, basta notare che Persona1 ha 365 giorni per nascere ed avere data diversa dalle altre persone, persona2 ne avrà 364, persona 3 ne avrà 363 e così via fino a persona25, la formula da utilizzare sarà quindi $\frac{365!}{(365-25)!}=\frac{k!}{(k-n)!}$, una volta calcolato $|A^c|$ e sapendo che $|\Omega|=365^{25}$ avremo $|A|=\frac{|A^c|}{365^{25}} =0,43130029...$, sembra un valore molto basso, ma in realtà $\probP$ è un valore compreso tra 0 e 1, questo equivarrebbe a circa il 43\%.

\section{Modello probabilistico}
abbiamo già definito non formalmente alcuni elementi del modello probabilistico. Il metodo probabilistico è formato da 3 elementi:
\begin{itemize}
    \item \textbf{Spazio degli eventi elementari} ($\boldsymbol{\Omega}$): l'insieme di tutti gli esiti possibili dell'esperimento
    
    \item \textbf{Algebra degli eventi} ($\boldsymbol{\mathcal{A}}$) è l'insieme chiuso di tutti i possibili sottoinsiemi di $\Omega$ ossia tutti i possibili eventi per i quali si può misurare la probabilità. Nel caso di un singolo esperimento, equivale all'insieme di tutti gli eventi che consideriamo soddisfacenti per l'esito dell'esperimento, nel caso del lancio di un dado non truccato a 6 facce $\Omega=\{1,2,3,4,5,6\}$, se lo scopo dell'esperimento è calcolare la probabilità che esca un valore superiore a 4, l'insieme $\mathcal{A}$ sarà formato dagli eventi dell'esperimento $\mathcal{A}=\{4,5,6\}$ ovvero gli eventi che alla domanda: \textit{l'evento risultante dall'esperimento, è superiore a 4?}
    
    \item \textbf{Probabilità ($\probP$)} è la funzione $\probP :\mathcal{A}\rightarrow [0,1]$ che associa ad ogni elemento dell'algebra degli eventi $\mathcal{A}$ una probabilità che viene misurata con un valore tra 0(evento impossibile) e 1(evento certo).
    la funzione \probP ha le seguenti proprietà:
    \begin{enumerate}
        \item \probP$(\emptyset)$ la probabilità di un insieme vuoto vale 0
        \item \probP$(\Omega)$ la probabilità di $\omega$ vale 1
        \item \probP  è una funzione additiva, quindi per due eventi separati $A$ e $B$, vale che $\probP(A\cup B)=\probP(A)+\probP(B)$. Per due eventi non separati, vale questa modifica $\probP(A\cup B)=\probP(A)+\probP(B)-\probP(A\cap B)$.
        \item Se $A \subseteq B$ allora $\probP(A) \le \probP(B)$
    \end{enumerate}
    è in fine necessario aggiungere che esiste anche una probabilità chiamata "probabilità degli eventi elementari" che ha come dominio $\Omega$ ed è la probabilità che accada un singolo evento, ad esempio che il lancio di un dado risulti in un 4.
\end{itemize}
    Dato che si lavora con degli insiemi, è necessario indicare le proprietà degli insiemi:
    \begin{itemize}
        \item \textbf{proprietà associativa}
            \begin{equation*}
                (A \cup B) \cup C = A \cup (B \cup C) \text{ e }(A \cap B) \cap C = A \cap (B \cap C)
            \end{equation*}
        
        \item \textbf{proprietà distributiva}
            \begin{equation*}
                (A \cup B) \cap C = (A \cap C) \cup (B \cap C)
            \end{equation*}
        
        \item \textbf{De morgan}
            \begin{equation*}
                (A^c \cup B^c) = (A \cap B)^c \text{ vale anche }(A^c \cap B^c) = (A \cup B)^c
            \end{equation*}
        
        \item \textbf{esclusione disgiuntiva}
            \begin{equation*}
                (A \cup B) - (A \cap B) = (A - B) \cup (B - A)
            \end{equation*}
        
    \end{itemize}

\section{Modello standard}
    come si calcola la probabilità di un evento A? per calcolare la probabilità di un evento si usa questa semplice formula già utilizzata nell'introduzione:
    
    \begin{equation*}
        \probP(A) = \frac{|A|}{|S|}
    \end{equation*}
    %
    \textbf{come ricavare $|A|$ e $|S|$}: per ricavare la cardinalità di S basta riuscire a calcolare il numero esatto di eventi in S, ovvero tutti gli eventi possibili nel nostro esperimento\\
    \textbf{esempio:} $|S|$ dei lanci di un dado a 6 facce è 
        \begin{equation*}
            S=\{1,2,3,4,5,6\} \text{ quindi }|S|=6
        \end{equation*} 
        %
        il calcolo di A cambia in base al'evento di cui vogliamo calcolare la probabilità, ma in sostanza significa contare il numero di tutti i gli eventi che soddisfacenti ad esempio nel calcolo della probabilità che lanciando 2 monete si avranno entrambe testa è $\frac{1}{4}$ infatti:
        \begin{align*}
            S=&{tt,cc,tc,ct} \\
            |S|=&4 \\
            A=& \{ tt \} \\
            |A|=& 1 \\
            \probP(A)=&\frac{|A|}{|S|}=\frac{1}{4}=0,25
        \end{align*}
        %
        facciamo un esempio un pò più complesso:\\
        si vuole trovare la probabilità che in un mazzo da 52 carte si trovino un asso o una carta con un determinato seme. Nel mazzo sono presenti 4 assi e 13 carte per ogni seme, è presente un asso per ogni seme.
        \begin{align*}
            S=&\{ \text{tutte le 52 carte} \} \\
            |S|=&52 \\
            B=&\{ \text{ evento in cui troviamo un asso} \} \\
            |B|=&4 \\
            C=& \{ \text{evento in cui troviamo la carta di un determinato seme} \}\\
            |C|=&13 \\
            \probP(A)=& \frac{|B\cup C|}{|S|}= \frac{(|B|+|C|)-(|B \cap C|)}{|S|}= \frac{(4+13)-1}{52} =\frac{16}{52}=\frac{4}{13}
        \end{align*}
        %
        Esistono alcuni casi in cui per comodità conviene calcolare il complementare dell'evento di cui vogliamo calcolare la probabilità e poi usare la formula
        %
        \begin{equation*}
            |A|= 1-\frac{A^c}{|S|}
        \end{equation*}
        %
        solitamente la si usa quando deve essere calcolata la probabilità che avvengano pochi eventi fra molti
\end{document}
