\documentclass{article}
\usepackage{graphicx, amsmath, amsfonts} %pacchetti aggiuntivi


\title{Algebra}
\author{Riccardo Cara}
\date{2023/2024}

\begin{document}

\maketitle
\tableofcontents
\newpage
\paragraph{requisiti}
    per questo corso sarà necessario avere esperienza con la teoria degli insiemi e le sue proprietà:
    \begin{itemize}

        \item intersezione
        \item unione
        \item sottoinsieme
        \item insieme complementare
        \item proprietà associativa
        \item proprietà distributiva
        \item De Morgan
        
    \end{itemize}
%############################################################################################
\section{Relazioni}
    Per capire le relazioni, occorre prima introdurre il prodotto cartesiano, ossia una tupla in cui ogni elemento $a \in A$ è associato ad un elemento $b \in B$ ovvero $A\times B=(a,b) | a \in A, b\in B$ ad esempio:
    se $A=\{1,2,3,4\}$ e $B=\{a,b,c,d\}$ il prodotto cartesiano $A\times B=\{(1,a),(2,b),(3,c),(4,d)\}$.
    Una relazione ($\rho$) non è altro che il sottoinsieme del prodotto cartesiano ovvero $\rho \subseteq A \times B$, se $(a,b)\in \rho$ si scrive $a\rho b$. Se una relazione è definita su A ossia $ a\rho a'|a \in A, a' \in A$ prende nome di \textbf{relazione di identità}. essendo la relazione un insieme, allora su essa valgono le proprietà degli insiemi.
    il dominio della relazione è:

    $\mathcal{D}= a \in A | \exists b\in B | a\rho b$ \\
    l'immagine della relazione è: 

    $\mathcal{I}=b \in B |\exists a \in A | a\rho b$ \\
    Se $\forall a \in A$ esiste un solo $b\in B | a\rho b$ allora $\rho$ è una funzione, ma non è detto che il suo inverso ($\rho^{-1}$) lo sia.
    ci sono 2 differenti modi di rappresentare graficamente le relazioni,

    \begin{tabular}{ c c c c c }
        \(d\) & 0 & 0 & 0 & 0  \\ 
        \(c\) & 0 & 1 & 1 & 0  \\ 
        \(b\) & 1 & 0 & 0 & 1  \\ 
        \(a\) & 0 & 0 & 1 & 0  \\ 
        &\(a\) & \(b\) & \(c\) & \(d\) 
    \end{tabular}

    rappresentazione tabellare
    diagramma 2%inserire diagramma 2
    rappresentazione con nodi e frecce.

    \subsection{Relazioni di equivalenza}
        Una relazione $\rho \in AxA$ è una \textit{relazione di equivalenza} se:

        \begin{itemize}
            \item \textbf{riflessiva}:$a\rho a \forall a \in A$
            \item \textbf{simmetrica}: se $a\rho a'$ allora esiste anche $a' \rho a$
            \item \textbf{transitiva}: se $a \rho a'$ e $a' \rho a''$ allora $a\rho a''$
        \end{itemize}

        con Le relazioni di equivalenza introduciamo anche \textit{le classi di equivalenza} scritte come $[a]=b\in A | a\rho b$ ovvero $[a]$ è l'insieme contenente tutti gli elementi in relazione con a (sono equivalenti ad a), avendo 2 classi di equivalenza $[a],[b] \in A$ se le due hanno almeno un elemento $c$ in comune allora $[a]=[b]$ poichè, se $c \in [a]$ significa che ogni elemento in $[a]$ è per definizione equivalente a $c$, stessa cosa per $[b]$, quindi $[a]$ e $[b]$ sono equivalenti.\\
        L'insieme delle classi di equivalenza in un insieme A viene detto \textit{insieme quoziente} e viene scritto come $A/a =[a] \forall [a] \in A$.

    \subsection{Partizioni}
        le partizioni($A_{\alpha}$) di un insieme ($A$) sono collezioni \textbf{diverse} di elementi dello stesso insieme tali che l'unione di essi risulti essere tutto l'insieme ($A_{\alpha}\cup A_{\beta}=A$) immaginare un diagramma a torta o semplicemente le partizioni di un HDD.
        le classi di equivalenza sono partizioni di $A$ essendo che le classi di equivalenza o sono congiunte (sono congiunte le classi $[a]=[b]$, nel diagramma a torta le classi [a]=[b] sono lo stesso spicchio) o disgiunte (spicchi diversi)

    \subsection{Relazioni di ordine parziale}
        una relazione è di ordine parziale quando è:

        \begin{itemize}
            \item \textbf{riflessiva} $a\rho a \forall a \in A$
            \item \textbf{antisimmetrica} dato $a\rho a'$ non si ha $a' \rho a$
            \item \textbf{Transitiva} se $a \rho a'$ e $a' \rho a''$ allora $a\rho a''$
        \end{itemize}

        un esempio di facile comprensione è la relazione tra sottoinsiemi, avendo $\mathcal{P}(X)$ ovvero un insieme composto da tutti i possibili sottoinsiemi di $X$ parti, definiamo la relazione $A\rho B|A\subseteq B$ abbiamo tutte le condizioni rispettate, infatti per ogni sottoinsieme di $\mathcal{P}(X)$ vale $A\rho A = A\subseteq A$, vale anche la seconda condizione, ovvero se $A\rho B$ e $B \rho C$ allora $A\rho C$ poichè se $A\subseteq B$ e $B\subseteq C$ allora $A\subseteq C$, è vera anche l'ultima condizione poichè $A\subseteq B$ implica che $B\not \subseteq A $ 

\section{insiemi e strutture algebriche}
    \subsection{numeri naturali}
        introduciamo un'astrazione dei numeri naturali ovvero la terna di Peano $(\mathbb{N},\sigma,0)$ e segue questi assiomi:

        \begin{itemize}
            \item esiste un numero $0 \in \mathbb{N}$
            
            \item $\sigma$ è una funzione $\sigma: \mathbb{N} \rightarrow \mathbb{N}$ chiamata successore
            
            \item $x\neq y$ implica $\sigma(x)\neq \sigma(y)$
            
            \item $\sigma(x)\neq 0 \forall x \in \mathbb{N}$
            
            \item se $U\subseteq \mathbb{N}$, $0\in U$, $x\in U$ e $\sigma(x)\in U$ allora $U=\mathbb{N}$ ovvero, ogni sottoinsieme di $\mathbb{N}$ che contiene lo $0$, e il successore di ogni numero nel sottoinsieme, coincide con $\mathbb{N}$
        \end{itemize}
%
        una volta definiti gli assiomi di Peano, possiamo definire delle operazioni elementari:

        \begin{itemize}
            
            \item\textbf{somma}
            definiamo la somma come un'operazione $\mathbb{N} \times \mathbb{N}\rightarrow \mathbb{N}$ ossia un'operazione che associa una oppia di elementi appartenenti all'insieme $\mathbb{N}$ ad un elemento dell'insieme $\mathbb{N}$, presi degli elementi $n,n',n''\in \mathbb{N}$ allora $n\times n'\rightarrow n''\equiv n+n'=n''$.

            possiamo notare nella somma che:
            \begin{itemize}
                \item $\sigma(n)+n'=\sigma(n+n')$
                \item $0+n=n$ poichè $0$ nella somma è un \textit{elemento neutro}\footnote{\textit{elemento neutro:} un elemento che non modifica nulla in un'operazione}
            \end{itemize}

            \item \textbf{prodotto}:
            definiamo il prodotto come l'operazione $\mathbb{N}\times \mathbb{N} \rightarrow \mathbb{N}$, presi gli elementi $n,n',n''\in \mathbb{N}$ allora $n\times n'\rightarrow n'' \equiv n\cdot n'=n''$

            possiamo notare nel prodotto che:
            \begin{itemize}
                \item $0\cdot n= 0 \forall n\in \mathbb{N}$
                \item $1\cdot n=n$ nell'operazione prodotto, $1$ è un elemento neutro
                \item $\sigma(n)\cdot n'= n\cdot n' + n''$ 
            \end{itemize}
        \end{itemize}
    \subsection{numeri interi}
        Una volta definiti i numeri naturali, visti il dominio e l'immagine, notiamo che non è possibile risolvere un'equazione come $x+1=0$, questo perchè il risultato non appartiene ai numeri naturali, bensì ai numeri interi $\mathbb{Z}$.
        È possibile definire i numeri interi partendo dai numeri naturali utilizzando l'insieme quoziente $\mathbb{Z}=\mathbb{N}\times \mathbb{N} /\sim$, definiamo con $n,m \in \mathbb{N}$ la relazione:

        \begin{equation}
            (n,m)\sim(n',m')\iff n+m'=m+n'
        \end{equation}
        
        ora prendiamo le coppie $(a,0),(0,a)$, $(a,0)$ è in relazione con tutte le coppie $n,m|n-m=a$ e $(0,a)$ è in relazione con tutte le coppie $n,m|n-m=-a$, per rendere la situazione più familiare, si possono associare i numeri che vengono in mente quando si pensa all'insieme dei numeri interi, come $-\infty...,-2,-1,0,1,2,...\infty$ alle coppie di valori $(a,0),(0,a)$. Associamo alla coppia $(a,0)$ i valori $a\in\mathbb{Z}|a\ge0$ e associamo alla coppia$(0,a)$ i valori $a\in\mathbb{Z}|a\le0$.
        Prima si è definito $\mathbb{Z}$ come un insieme quoziente, un insieme quoziente è l'insieme delle classi di equivalenza di un insieme e le classi di equivalenza dell'insieme che stiamo analizzando sono $[(n,m)]$.
        Definiamo le operazioni:

        \begin{itemize}
            \item \textbf{somma}: $[(n,m)]+[(n',m')]=[(n+n',m+m')]$ ad esempio, la somma $[(0,2)]+[(5,0)]=[(5,2)]=[(3,0)]=3$, un modo di facile e veloce di trovare la classe di equivalenza in formato $[(a,0)]$ o $[(0,a)]$ è semplicemente, sostituire il valore più piccolo della coppia ($min$) con $0$ e sostituire il valore più grande della coppia ($MAX$) con $MAX-min$
            \item \textbf{prodotto}:$[(n,m)]\cdot[(n',m')]=[(n\cdot n'+m\cdot m', m\cdot n'+n\cdot m')]$, ad esempio, il prodotto $[(0,2)]\cdot[(5,0)]=[(0\cdot5+2\cdot0,2\cdot5+0\cdot 0)]=[(0,10)]=-10$
        \end{itemize}
    
    \subsection{Divisibilità in $\mathbb{Z}$}
        presi 2 numeri $a,b\in\mathbb{Z}$ con $b\neq 0$esistono solo due numeri unici $q,r\in\mathbb{Z}$ tali che:

        \begin{equation}
        a=bq+r, 0\leq r \leq |b|
        \end{equation}
        
        diciamo che:
        
        \begin{itemize}
            \item $a|b$ si dice a divide $b$ se $\exists c \in \mathbb{Z}:b=ac$
            \item $a|0\forall a\in \mathbb{Z}$
            \item ogni $a\in \mathbb{Z}$ ha divisori $\pm1,\pm a$
            \item $0|a \iff a=0$
            \item $a|1\iff a=\pm1$
            \item se $a|b$ e $a|c$ allora $a|bx+cy, \forall x,\forall y$ e viceversa
        \end{itemize}
    
    \subsection{MCD}
        siano $a,b\in \mathbb{Z}$, $d\geq 1$ si dice MCD se $d|a$ e $d|b$
        per calcolare il MCD si utilizza l'algoritmo Euclideo:

        \begin{enumerate}
            \item si divide a per b, si ottengono $q_1$ e $r_1$, se $r_1\neq 0$ si continua
            \item si divide b per $r_1$, si ottengono $q_2$ e $r_2$, se $r_2\neq0$ si continua
            \item si divide $r_1$ per $r_2$, si ottengono $q_3$ e $r_3$, se $r_3\neq0$ si continua
            \item[n.] si divide $r_{n-2}$ per $r_{n-1}$, si ottengono $q_n$ e $r_n$, se $r_n\neq0$ si continua
            \item[n+1.] si divide $r_{n-1}$ per $r_n$, si ottengono $q_{n+1}$ e $r_{n+1}$, a questo punto, $r_{n+1}=0$ e l'$MCD(a,b)=r_n$ ovvero l'ultimo $r\neq0$
        \end{enumerate}

    \subsection{Equazioni diofantee}

    \subsection{minimo comune multiplo}
        il minimo comune multiplo, indicato come mcm(a,b) è il valore $k\geq 0:a|h, b|h$.
        se $a,b\neq 0$ e $a,b\notin \mathbb{Z}$ allora $|ab|= MCD(a,b)\cdot mcm(a,b)$, ne ricaviamo $mcm=\frac{|ab|}{MCD(a,b)}$
    \subsection{Numeri primi}
        i numeri primi sono numeri natruali maggiori di 1 divisibili solo da 1 e da se stessi, dato un numero naturale maggiore di 1, si scrivono tutti i sottomultipli (o divisori) $D(5)=\{1,5\}$, $D(4)=\{1,2,4\}$, $D(15)=\{1,3,5,15\}$, $D(13)=\{1,13\}$ si può constatare che 5 e 13 sono numeri primi p

    \subsection{Teorema fondamentale dell'aritmentica}
        il teorema fondamentale del'aritmetica afferma che un numero $n\geq 2\in \mathbb{N}$ è un numero primo o si può scrivere come prodotto di numeri primi. Il numero n è un numero composto dal prodotto:

        \begin{equation}
            n=p_1^{e_1}\cdot p_2^{e_2}\cdot ...\cdot p_n^{e_n}, p_n\geq 1,e_n\geq 1
        \end{equation}
         %
        $p_1,p_2,p_3$ sono numeri primi diversi (2,3,5,7,9,...), ad esempio $28=2*2*7=2^2*7^1$

\section{Strutture algebriche notevoli}
    enunciamo una definizione necessaria per la comprensione dei prossimi argomenti. Sia $X$ un insieme, è possibile definire su esso un'operazione binaria $X\times X \rightarrow X$ chiamata applicazione. l'insieme $(\mathbb{Z},+)$ è un insieme composto da numeri interi con l'operazione binaria "+" definita su esso.

    \subsection{Semigruppo}
        un semigruppo è un insieme S dotato di un'operazione binaria $\ast$ con le seguenti proprietà:

        \begin{itemize}
            \item $\ast$ è associativa, $(s\ast s')\ast s''= s\ast s'\ast s''$
            \item $\exists e\in S | s\ast e = s = e\ast s\forall s \in S$, ovvero e è un elemento nullo, l'elemento nullo è unico
        \end{itemize}
        %
        esiste anche il semigruppo commutativo, ovvero un semigruppo in cui oltre alle proprietà elencate, si ha che $s\ast s'= s' \ast s\forall s \in S$.

    \subsection{Gruppo}
        un gruppo è un insieme G dotato di un operazione binaria $\ast$ con le seguenti proprietà:

        \begin{itemize}
            \item $\ast$ è associativa, $(g\ast g')\ast g''= g\ast g'\ast g''$
            \item $\exists e\in G | g\ast e = g = e\ast g \forall g \in G$, ovvero è un elemento nullo, l'elemento nullo è unico
            \item $\forall g \in G \exists g' |g\ast g'=e=g'\ast g$, ovvero per ogni elemento s, vi è  il suo inverso, se moltiplicati tra loro viene restituito l'elemento nullo.
        \end{itemize}
%
        un esempio di gruppo è $(\mathbb{Z},+)$ poichè $\forall z \in \mathbb{Z}\exists -z|z+(-z)=0$

    \subsection{Anello}
        Un anello $(A,\odot,\ast)$ è un insieme avente 2 operazioni binarie aventi le seguenti proprietà:
        \begin{itemize}
            \item $(A,\odot)$ è un gruppo commutativo, l'elemento neutro è $O_A$
            \item $\ast$ è associativa, ossia $(a\ast a')\ast a''=a\ast a' \ast a''$
            \item vale la proprietà distributiva: $(a\odot a')\ast a''=(a\ast a'')\odot(a'\ast a'')$
        \end{itemize}
        
        si dice anello commutativo un anello in cui anche l'operazione $\ast$ è commutativa.
        si dice anello unitario, un anello che ha un elemento neutro anche sull'operazione $\ast$, ossia $\exists u \in A | a \ast u = a = u \ast a\forall a\in A$, u è un unità.
        Se un anello commutativo è unitario ed è privo di divisori dello zero (ovvero $a\ast b = O_a \Rightarrow a=O_a\lor b=O_A$) viene detto dominio di integrità, l'insieme dei numeri interi $(\mathbb{Z},+,\cdot,0)$ è un dominio di integrità.
        \textbf{proprietà}

        \begin{itemize}
            \item $\forall a \in A, a\ast 0 = 0$
            \item $a \ast (-a')= (-aa')=(-a)\ast a'$
            \item $(-a)\ast (-a')=aa'$
        \end{itemize}

    \subsection{campo}
        Un campo è un Anello commutativo unitario in cui $\forall k\neq 0\in \mathbb{K}$ ha il proprio inverso.
    
    \subsection{anello $\mathbb{Z}_n$}
        l'anello $\mathbb{Z}_n\equiv \mathbb{Z}/_{\sim_n}$(insieme quoziente) è l'anello commutativo unitario con divisori dello zero (quindi non è un dominio di integrità). Definiamo $\sim_n$ come la relazione 
        
        \begin{equation}
            a \sim_n b \Leftrightarrow a-b \text{ è divisibile per } n
        \end{equation}
%
        sappiamo quindi che essendo $\mathbb{Z}$ l'insieme quoziente è l'insieme di tutte le classi di equivalenza ovvero $\mathbb{Z}_n=\{[0],[1],...,[n-1]\}$, su tale insieme sono definite somma e prodotto.

        \begin{equation}
            [z]+[z']=[z+z'] \text{ e } [z]\cdot[z']=[z\cdot z'] 
        \end{equation}
        %
        si osserva che, l'anello $\mathbb{Z}_n$ e commutativo, unitario (poichè ha elemento neutro per la somma e=[0] e elemento neutro per il prodotto u=[1]) ed ha anche divisori dello zero ovvero si infrange la regola $a\cdot b =0 \Leftrightarrow a=0\lor b=0$, poichè $[4]\cdot[3]=[12]$ e $[12]=[0]$ infatti $12\sim_{12} 0 \Leftrightarrow 12-0$ è divisibile per $12$, siccome 12 e 0 sono in relazione, l'insieme di equivalenza è identico, quindi $[12]=[0]$.
    
    \subsection{congruenze}
        Dati gli interi $a,b,m$ si dice che a e b sono congruenti quando $a\equiv b (\text{mod } m) \Leftrightarrow \frac{a}{m}=\frac{b}{m}$ ovvero quando a e b hanno lo stesso resto se divisi per m, ad esempio $48\equiv 3 (\text{mod }5)$ perchè $\frac{48}{5}=9$con resto 3 $\frac{3}{5}=0$ con resto 3
\end{document}
